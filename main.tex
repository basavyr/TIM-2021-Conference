\documentclass{beamer}

\mode<presentation> {
\usetheme{Madrid}
}

\usepackage{graphicx} % Allows including images
\usepackage{booktabs} % Allows the use of \toprule, \midrule and \bottomrule in tables}
\usepackage{amsmath}
\usepackage{xcolor}
\usepackage{physics}
%----------------------------------------------------------------------------------------
%	TITLE PAGE
%----------------------------------------------------------------------------------------

\title[Wobbling Motion in odd-A nuclei]{A Novel Approach for the Semi-Classical Description of the Wobbling Properties in Odd-A Nuclei} % The short title appears at the bottom of every slide, the full title is only on the title page

\author{Robert Poenaru} % Your name
\institute[DFT] % Your institution as it will appear on the bottom of every slide, may be shorthand to save space
{
Department of Theoretical Physics, IFIN-HH \\
\vspace{0.1cm}
Faculty of Physics, University of Bucharest \\% Your institution for the title page
\medskip
\textit{robert.poenaru@drd.unibuc.ro} % Your email address
}
\date{\today} % Date, can be changed to a custom date

\begin{document}

\begin{frame}
\titlepage % Print the title page as the first slide
\end{frame}

\begin{frame}
\frametitle{Outline} % Table of contents slide, comment this block out to remove it
\tableofcontents % Throughout your presentation, if you choose to use \section{} and \subsection{} commands, these will automatically be printed on this slide as an overview of your presentation
\end{frame}


\section{Wobbling Motion} 

\begin{frame}{Shape parametrization}
Nuclear shapes can be described via the \textbf{quadrupole parameter} $\beta$ and the \textbf{triaxiality parameter} $\gamma$.
\begin{figure}
    \centering
    \includegraphics[scale=0.5]{figs/beta_gamma_plane.png}
    \caption{The $(\beta,\gamma$) plane divided into six equivalent parts.}
    \label{fig:betagamma}
\end{figure}
\end{frame}

\begin{frame}{Triaxiality}
Nuclear shapes: most of the nuclei are spherical or axially symmetric in the ground state.
  \begin{figure}
    \centering
    \includegraphics[scale=0.4]{figs/nuclear_shapes.png}
    \caption{\textbf{Spherical:} $\beta_2=0$ ; \textbf{Prolate:} $\beta_2>0$ ; \textbf{Oblate:} $\beta_2<0$}
  \end{figure}
  \begin{block}{Triaxial shapes}
       There are also deviations from \emph{axial symmetric shapes} $\to$ \textbf{triaxial shapes} (e.g. no symmetry axis). The three PA's have different lengths (asymmetry between the MOIs).
  \end{block}
\end{frame}

\begin{columns}
    \begin{column}{0.5\textwidth}
    \begin{block}{Wobbling motion (WM)}
  \begin{itemize}
    \item Uniquely associated to triaxial structures. 
    \item It was theoretically predicted by Bohr and Mottelson more than 50 years ago (for the even-$A$ case).
    \item Experimentally confirmed for $^{163}$Lu in 2001.
  \end{itemize}
  \end{block}
  \begin{exampleblock}{Experimental evidence}
  In present, few wobblers are experimentally confirmed in the mass regions: $A\approx130,160,180$.
  \end{exampleblock}
  \end{column}
  \begin{column}{0.5\textwidth}
        \begin{figure}
          \centering
          \includegraphics[scale=0.35]{figs/wobbling_drawing.png}
          \caption{Schematic representation for the nuclear wobbling motion.}
          \label{wobbling_picture}
      \end{figure}
  \end{column}
  \end{columns}
\end{frame}


% \begin{frame}
% \frametitle{Multiple Columns}
% \begin{columns}[c] % The "c" option specifies centered vertical alignment while the "t" option is used for top vertical alignment

% \column{.45\textwidth} % Left column and width
% \textbf{Heading}
% \begin{enumerate}
% \item Statement
% \item Explanation
% \item Example
% \end{enumerate}

% \column{.5\textwidth} % Right column and width
% Lorem ipsum dolor sit amet, consectetur adipiscing elit. Integer lectus nisl, ultricies in feugiat rutrum, porttitor sit amet augue. Aliquam ut tortor mauris. Sed volutpat ante purus, quis accumsan dolor.

% \end{columns}
% \end{frame}

% \begin{frame}
% \frametitle{References}
% \footnotesize{
% \begin{thebibliography}{99} % Beamer does not support BibTeX so references must be inserted manually as below
% \bibitem[Smith, 2012]{p1} John Smith (2012)
% \newblock Title of the publication
% \newblock \emph{Journal Name} 12(3), 45 -- 678.
% \end{thebibliography}
% }
% \end{frame}

\begin{frame}
\Huge{\centerline{Thank you for your attention!}}
\end{frame}

\end{document} 